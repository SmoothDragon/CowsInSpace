
    % latexmk -pdflatex='lualatex' -pdf FontExhibition.tex
    \documentclass[parskip,landscape,letter]{scrartcl}
    \usepackage{fontspec}
    \usepackage[margin=10mm,left=20mm]{geometry}
    \usepackage{tikz}
    \usetikzlibrary{matrix,backgrounds}
    \usepackage{pifont}
    \usepackage{graphicx}
    \usepackage{chessfss}
    \usepackage{setspace}
    \usepackage{enumitem}
    \usepackage{amssymb}
    \usepackage{graphicx,calc}
    \graphicspath{{./graphics/}}
    \input{symbols.tex}

    \begin{document}
    \setmainfont[Extension={.ttf},ItalicFont={DejaVuSerif-Italic}]{FreeSerif}
    
\begin{tikzpicture}
    \pgfmathsetmacro{\cardroundingradius}{5mm}
    \pgfmathsetmacro{\striproundingradius}{3mm}
    % \pgfmathsetmacro{\cardwidth}{5.9}
    % \pgfmathsetmacro{\cardheight}{9.2}
    \pgfmathsetmacro{\cardwidth}{6.1}  % Magic cards are 63x88mm
    \pgfmathsetmacro{\cardheight}{8.6}
    \pgfmathsetmacro{\stripwidth}{1.2}
    \pgfmathsetmacro{\strippadding}{0.1}
    \pgfmathsetmacro{\textpadding}{0.3}
    \pgfmathsetmacro{\ruleheight}{0.1}
    \providecommand{\stripfontsize}{\Huge}
    \providecommand{\captionfontsize}{\LARGE}
    \providecommand{\textfontsize}{\Large}
    \providecommand{\quotefontsize}{\small}
    \draw[line width=2mm,rounded corners=\cardroundingradius] (0,0) rectangle (\cardwidth,\cardheight);
    \draw[line width=2mm] (0,0) rectangle (\cardwidth,\cardheight);
    \node[text width=(\cardwidth-\strippadding-2*\textpadding)*1cm,below right,inner sep=0] at (\strippadding+\textpadding,\cardheight-\textpadding) 
    { 
    \begin{center} {\fontsize{80pt}{60pt}\selectfont \COW}\\\end{center}
\begin{center}
    {\captionfontsize \textsf{\textbf{LOW ENERGY STATE}}}\end{center}
        {\textfontsize Non-designated cows move only 1 hex at a time.}
        \tikz{\fill (0,0) rectangle (\cardwidth-2*\strippadding-2*\textpadding,\ruleheight);}\\
        {\quotefontsize \textit{}}\\[-2\baselineskip]
    };
    \node[circle,draw,text=black](c) at (.5,\cardheight-.5){};
    \node[circle,draw,text=black](c) at (\cardwidth-.5,\cardheight-.5){};
    \end{tikzpicture}%
\hspace{1pt}%
\begin{tikzpicture}
    \pgfmathsetmacro{\cardroundingradius}{5mm}
    \pgfmathsetmacro{\striproundingradius}{3mm}
    % \pgfmathsetmacro{\cardwidth}{5.9}
    % \pgfmathsetmacro{\cardheight}{9.2}
    \pgfmathsetmacro{\cardwidth}{6.1}  % Magic cards are 63x88mm
    \pgfmathsetmacro{\cardheight}{8.6}
    \pgfmathsetmacro{\stripwidth}{1.2}
    \pgfmathsetmacro{\strippadding}{0.1}
    \pgfmathsetmacro{\textpadding}{0.3}
    \pgfmathsetmacro{\ruleheight}{0.1}
    \providecommand{\stripfontsize}{\Huge}
    \providecommand{\captionfontsize}{\LARGE}
    \providecommand{\textfontsize}{\Large}
    \providecommand{\quotefontsize}{\small}
    \draw[line width=2mm,rounded corners=\cardroundingradius] (0,0) rectangle (\cardwidth,\cardheight);
    \draw[line width=2mm] (0,0) rectangle (\cardwidth,\cardheight);
    \node[text width=(\cardwidth-\strippadding-2*\textpadding)*1cm,below right,inner sep=0] at (\strippadding+\textpadding,\cardheight-\textpadding) 
    { 
    \begin{center} {\fontsize{80pt}{60pt}\selectfont \COW}\\\end{center}
\begin{center}
    {\captionfontsize \textsf{\textbf{TARGET ORBITALS}}}\end{center}
        {\textfontsize Cows may also orbit target hexes.}
        \tikz{\fill (0,0) rectangle (\cardwidth-2*\strippadding-2*\textpadding,\ruleheight);}\\
        {\quotefontsize \textit{}}\\[-2\baselineskip]
    };
    \node[circle,draw,text=black](c) at (.5,\cardheight-.5){};
    \node[circle,draw,text=black](c) at (\cardwidth-.5,\cardheight-.5){};
    \end{tikzpicture}%
\hspace{1pt}%
\begin{tikzpicture}
    \pgfmathsetmacro{\cardroundingradius}{5mm}
    \pgfmathsetmacro{\striproundingradius}{3mm}
    % \pgfmathsetmacro{\cardwidth}{5.9}
    % \pgfmathsetmacro{\cardheight}{9.2}
    \pgfmathsetmacro{\cardwidth}{6.1}  % Magic cards are 63x88mm
    \pgfmathsetmacro{\cardheight}{8.6}
    \pgfmathsetmacro{\stripwidth}{1.2}
    \pgfmathsetmacro{\strippadding}{0.1}
    \pgfmathsetmacro{\textpadding}{0.3}
    \pgfmathsetmacro{\ruleheight}{0.1}
    \providecommand{\stripfontsize}{\Huge}
    \providecommand{\captionfontsize}{\LARGE}
    \providecommand{\textfontsize}{\Large}
    \providecommand{\quotefontsize}{\small}
    \draw[line width=2mm,rounded corners=\cardroundingradius] (0,0) rectangle (\cardwidth,\cardheight);
    \draw[line width=2mm] (0,0) rectangle (\cardwidth,\cardheight);
    \node[text width=(\cardwidth-\strippadding-2*\textpadding)*1cm,below right,inner sep=0] at (\strippadding+\textpadding,\cardheight-\textpadding) 
    { 
    \begin{center} {\fontsize{80pt}{60pt}\selectfont \COW}\\\end{center}
\begin{center}
    {\captionfontsize \textsf{\textbf{EMERGENCY BRAKE}}}\end{center}
        {\textfontsize Cows may stop anywhere, but cannot move again afterwards.}
        \tikz{\fill (0,0) rectangle (\cardwidth-2*\strippadding-2*\textpadding,\ruleheight);}\\
        {\quotefontsize \textit{}}\\[-2\baselineskip]
    };
    \node[circle,draw,text=black](c) at (.5,\cardheight-.5){};
    \node[circle,draw,text=black](c) at (\cardwidth-.5,\cardheight-.5){};
    \end{tikzpicture}%
\hspace{1pt}%
\begin{tikzpicture}
    \pgfmathsetmacro{\cardroundingradius}{5mm}
    \pgfmathsetmacro{\striproundingradius}{3mm}
    % \pgfmathsetmacro{\cardwidth}{5.9}
    % \pgfmathsetmacro{\cardheight}{9.2}
    \pgfmathsetmacro{\cardwidth}{6.1}  % Magic cards are 63x88mm
    \pgfmathsetmacro{\cardheight}{8.6}
    \pgfmathsetmacro{\stripwidth}{1.2}
    \pgfmathsetmacro{\strippadding}{0.1}
    \pgfmathsetmacro{\textpadding}{0.3}
    \pgfmathsetmacro{\ruleheight}{0.1}
    \providecommand{\stripfontsize}{\Huge}
    \providecommand{\captionfontsize}{\LARGE}
    \providecommand{\textfontsize}{\Large}
    \providecommand{\quotefontsize}{\small}
    \draw[line width=2mm,rounded corners=\cardroundingradius] (0,0) rectangle (\cardwidth,\cardheight);
    \draw[line width=2mm] (0,0) rectangle (\cardwidth,\cardheight);
    \node[text width=(\cardwidth-\strippadding-2*\textpadding)*1cm,below right,inner sep=0] at (\strippadding+\textpadding,\cardheight-\textpadding) 
    { 
    \begin{center} {\fontsize{80pt}{60pt}\selectfont \COW}\\\end{center}
\begin{center}
    {\captionfontsize \textsf{\textbf{FORCE FIELD}}}\end{center}
        {\textfontsize Cows may turn on a force field turning the surrounding hexes into walls.}
        \tikz{\fill (0,0) rectangle (\cardwidth-2*\strippadding-2*\textpadding,\ruleheight);}\\
        {\quotefontsize \textit{}}\\[-2\baselineskip]
    };
    \node[circle,draw,text=black](c) at (.5,\cardheight-.5){};
    \node[circle,draw,text=black](c) at (\cardwidth-.5,\cardheight-.5){};
    \end{tikzpicture}%
\hspace{1pt}%
\\[-.5\lineskip]
\begin{tikzpicture}
    \pgfmathsetmacro{\cardroundingradius}{5mm}
    \pgfmathsetmacro{\striproundingradius}{3mm}
    % \pgfmathsetmacro{\cardwidth}{5.9}
    % \pgfmathsetmacro{\cardheight}{9.2}
    \pgfmathsetmacro{\cardwidth}{6.1}  % Magic cards are 63x88mm
    \pgfmathsetmacro{\cardheight}{8.6}
    \pgfmathsetmacro{\stripwidth}{1.2}
    \pgfmathsetmacro{\strippadding}{0.1}
    \pgfmathsetmacro{\textpadding}{0.3}
    \pgfmathsetmacro{\ruleheight}{0.1}
    \providecommand{\stripfontsize}{\Huge}
    \providecommand{\captionfontsize}{\LARGE}
    \providecommand{\textfontsize}{\Large}
    \providecommand{\quotefontsize}{\small}
    \draw[line width=2mm,rounded corners=\cardroundingradius] (0,0) rectangle (\cardwidth,\cardheight);
    \draw[line width=2mm] (0,0) rectangle (\cardwidth,\cardheight);
    \node[text width=(\cardwidth-\strippadding-2*\textpadding)*1cm,below right,inner sep=0] at (\strippadding+\textpadding,\cardheight-\textpadding) 
    { 
    \begin{center} {\fontsize{80pt}{60pt}\selectfont \COW}\\\end{center}
\begin{center}
    {\captionfontsize \textsf{\textbf{INELASTIC COLLISION}}}\end{center}
        {\textfontsize A cow may push another cow any number of hexes before stopping.}
        \tikz{\fill (0,0) rectangle (\cardwidth-2*\strippadding-2*\textpadding,\ruleheight);}\\
        {\quotefontsize \textit{}}\\[-2\baselineskip]
    };
    \node[circle,draw,text=black](c) at (.5,\cardheight-.5){};
    \node[circle,draw,text=black](c) at (\cardwidth-.5,\cardheight-.5){};
    \end{tikzpicture}%
\hspace{1pt}%
\begin{tikzpicture}
    \pgfmathsetmacro{\cardroundingradius}{5mm}
    \pgfmathsetmacro{\striproundingradius}{3mm}
    % \pgfmathsetmacro{\cardwidth}{5.9}
    % \pgfmathsetmacro{\cardheight}{9.2}
    \pgfmathsetmacro{\cardwidth}{6.1}  % Magic cards are 63x88mm
    \pgfmathsetmacro{\cardheight}{8.6}
    \pgfmathsetmacro{\stripwidth}{1.2}
    \pgfmathsetmacro{\strippadding}{0.1}
    \pgfmathsetmacro{\textpadding}{0.3}
    \pgfmathsetmacro{\ruleheight}{0.1}
    \providecommand{\stripfontsize}{\Huge}
    \providecommand{\captionfontsize}{\LARGE}
    \providecommand{\textfontsize}{\Large}
    \providecommand{\quotefontsize}{\small}
    \draw[line width=2mm,rounded corners=\cardroundingradius] (0,0) rectangle (\cardwidth,\cardheight);
    \draw[line width=2mm] (0,0) rectangle (\cardwidth,\cardheight);
    \node[text width=(\cardwidth-\strippadding-2*\textpadding)*1cm,below right,inner sep=0] at (\strippadding+\textpadding,\cardheight-\textpadding) 
    { 
    \begin{center} {\fontsize{80pt}{60pt}\selectfont \COW}\\\end{center}
\begin{center}
    {\captionfontsize \textsf{\textbf{QUANTUM TUNNELING}}}\end{center}
        {\textfontsize A cow may move through one blockade and continue moving in the same direction.}
        \tikz{\fill (0,0) rectangle (\cardwidth-2*\strippadding-2*\textpadding,\ruleheight);}\\
        {\quotefontsize \textit{}}\\[-2\baselineskip]
    };
    \node[circle,draw,text=black](c) at (.5,\cardheight-.5){};
    \node[circle,draw,text=black](c) at (\cardwidth-.5,\cardheight-.5){};
    \end{tikzpicture}%
\hspace{1pt}%
\begin{tikzpicture}
    \pgfmathsetmacro{\cardroundingradius}{5mm}
    \pgfmathsetmacro{\striproundingradius}{3mm}
    % \pgfmathsetmacro{\cardwidth}{5.9}
    % \pgfmathsetmacro{\cardheight}{9.2}
    \pgfmathsetmacro{\cardwidth}{6.1}  % Magic cards are 63x88mm
    \pgfmathsetmacro{\cardheight}{8.6}
    \pgfmathsetmacro{\stripwidth}{1.2}
    \pgfmathsetmacro{\strippadding}{0.1}
    \pgfmathsetmacro{\textpadding}{0.3}
    \pgfmathsetmacro{\ruleheight}{0.1}
    \providecommand{\stripfontsize}{\Huge}
    \providecommand{\captionfontsize}{\LARGE}
    \providecommand{\textfontsize}{\Large}
    \providecommand{\quotefontsize}{\small}
    \draw[line width=2mm,rounded corners=\cardroundingradius] (0,0) rectangle (\cardwidth,\cardheight);
    \draw[line width=2mm] (0,0) rectangle (\cardwidth,\cardheight);
    \node[text width=(\cardwidth-\strippadding-2*\textpadding)*1cm,below right,inner sep=0] at (\strippadding+\textpadding,\cardheight-\textpadding) 
    { 
    \begin{center} {\fontsize{80pt}{60pt}\selectfont \COW}\\\end{center}
\begin{center}
    {\captionfontsize \textsf{\textbf{SINGLE INSTRUCTION, MULTIPLE DATA}}}\end{center}
        {\textfontsize The same movement instruction may be given to multiple cows at the same time.}
        \tikz{\fill (0,0) rectangle (\cardwidth-2*\strippadding-2*\textpadding,\ruleheight);}\\
        {\quotefontsize \textit{}}\\[-2\baselineskip]
    };
    \node[circle,draw,text=black](c) at (.5,\cardheight-.5){};
    \node[circle,draw,text=black](c) at (\cardwidth-.5,\cardheight-.5){};
    \end{tikzpicture}%
\hspace{1pt}%
\begin{tikzpicture}
    \pgfmathsetmacro{\cardroundingradius}{5mm}
    \pgfmathsetmacro{\striproundingradius}{3mm}
    % \pgfmathsetmacro{\cardwidth}{5.9}
    % \pgfmathsetmacro{\cardheight}{9.2}
    \pgfmathsetmacro{\cardwidth}{6.1}  % Magic cards are 63x88mm
    \pgfmathsetmacro{\cardheight}{8.6}
    \pgfmathsetmacro{\stripwidth}{1.2}
    \pgfmathsetmacro{\strippadding}{0.1}
    \pgfmathsetmacro{\textpadding}{0.3}
    \pgfmathsetmacro{\ruleheight}{0.1}
    \providecommand{\stripfontsize}{\Huge}
    \providecommand{\captionfontsize}{\LARGE}
    \providecommand{\textfontsize}{\Large}
    \providecommand{\quotefontsize}{\small}
    \draw[line width=2mm,rounded corners=\cardroundingradius] (0,0) rectangle (\cardwidth,\cardheight);
    \draw[line width=2mm] (0,0) rectangle (\cardwidth,\cardheight);
    \node[text width=(\cardwidth-\strippadding-2*\textpadding)*1cm,below right,inner sep=0] at (\strippadding+\textpadding,\cardheight-\textpadding) 
    { 
    \begin{center} {\fontsize{80pt}{60pt}\selectfont \COW}\\\end{center}
\begin{center}
    {\captionfontsize \textsf{\textbf{TARGET TURNER}}}\end{center}
        {\textfontsize Any number of targets may be rotated for a turn.}
        \tikz{\fill (0,0) rectangle (\cardwidth-2*\strippadding-2*\textpadding,\ruleheight);}\\
        {\quotefontsize \textit{}}\\[-2\baselineskip]
    };
    \node[circle,draw,text=black](c) at (.5,\cardheight-.5){};
    \node[circle,draw,text=black](c) at (\cardwidth-.5,\cardheight-.5){};
    \end{tikzpicture}%
\hspace{1pt}%
\\[-.5\lineskip]
\begin{tikzpicture}
    \pgfmathsetmacro{\cardroundingradius}{5mm}
    \pgfmathsetmacro{\striproundingradius}{3mm}
    % \pgfmathsetmacro{\cardwidth}{5.9}
    % \pgfmathsetmacro{\cardheight}{9.2}
    \pgfmathsetmacro{\cardwidth}{6.1}  % Magic cards are 63x88mm
    \pgfmathsetmacro{\cardheight}{8.6}
    \pgfmathsetmacro{\stripwidth}{1.2}
    \pgfmathsetmacro{\strippadding}{0.1}
    \pgfmathsetmacro{\textpadding}{0.3}
    \pgfmathsetmacro{\ruleheight}{0.1}
    \providecommand{\stripfontsize}{\Huge}
    \providecommand{\captionfontsize}{\LARGE}
    \providecommand{\textfontsize}{\Large}
    \providecommand{\quotefontsize}{\small}
    \draw[line width=2mm,rounded corners=\cardroundingradius] (0,0) rectangle (\cardwidth,\cardheight);
    \draw[line width=2mm] (0,0) rectangle (\cardwidth,\cardheight);
    \node[text width=(\cardwidth-\strippadding-2*\textpadding)*1cm,below right,inner sep=0] at (\strippadding+\textpadding,\cardheight-\textpadding) 
    { 
    \begin{center} {\fontsize{80pt}{60pt}\selectfont \COW}\\\end{center}
\begin{center}
    {\captionfontsize \textsf{\textbf{WIDE REACH}}}\end{center}
        {\textfontsize A cow may stop if either of the side diagonals are blocked.}
        \tikz{\fill (0,0) rectangle (\cardwidth-2*\strippadding-2*\textpadding,\ruleheight);}\\
        {\quotefontsize \textit{}}\\[-2\baselineskip]
    };
    \node[circle,draw,text=black](c) at (.5,\cardheight-.5){};
    \node[circle,draw,text=black](c) at (\cardwidth-.5,\cardheight-.5){};
    \end{tikzpicture}%
\hspace{1pt}%
\begin{tikzpicture}
    \pgfmathsetmacro{\cardroundingradius}{5mm}
    \pgfmathsetmacro{\striproundingradius}{3mm}
    % \pgfmathsetmacro{\cardwidth}{5.9}
    % \pgfmathsetmacro{\cardheight}{9.2}
    \pgfmathsetmacro{\cardwidth}{6.1}  % Magic cards are 63x88mm
    \pgfmathsetmacro{\cardheight}{8.6}
    \pgfmathsetmacro{\stripwidth}{1.2}
    \pgfmathsetmacro{\strippadding}{0.1}
    \pgfmathsetmacro{\textpadding}{0.3}
    \pgfmathsetmacro{\ruleheight}{0.1}
    \providecommand{\stripfontsize}{\Huge}
    \providecommand{\captionfontsize}{\LARGE}
    \providecommand{\textfontsize}{\Large}
    \providecommand{\quotefontsize}{\small}
    \draw[line width=2mm,rounded corners=\cardroundingradius] (0,0) rectangle (\cardwidth,\cardheight);
    \draw[line width=2mm] (0,0) rectangle (\cardwidth,\cardheight);
    \node[text width=(\cardwidth-\strippadding-2*\textpadding)*1cm,below right,inner sep=0] at (\strippadding+\textpadding,\cardheight-\textpadding) 
    { 
    \begin{center} {\fontsize{80pt}{60pt}\selectfont \COW}\\\end{center}
\begin{center}
    {\captionfontsize \textsf{\textbf{TELEPORT}}}\end{center}
        {\textfontsize Non-designated cows may use a move to telport from one target to another.}
        \tikz{\fill (0,0) rectangle (\cardwidth-2*\strippadding-2*\textpadding,\ruleheight);}\\
        {\quotefontsize \textit{}}\\[-2\baselineskip]
    };
    \node[circle,draw,text=black](c) at (.5,\cardheight-.5){};
    \node[circle,draw,text=black](c) at (\cardwidth-.5,\cardheight-.5){};
    \end{tikzpicture}%
\hspace{1pt}%
\begin{tikzpicture}
    \pgfmathsetmacro{\cardroundingradius}{5mm}
    \pgfmathsetmacro{\striproundingradius}{3mm}
    % \pgfmathsetmacro{\cardwidth}{5.9}
    % \pgfmathsetmacro{\cardheight}{9.2}
    \pgfmathsetmacro{\cardwidth}{6.1}  % Magic cards are 63x88mm
    \pgfmathsetmacro{\cardheight}{8.6}
    \pgfmathsetmacro{\stripwidth}{1.2}
    \pgfmathsetmacro{\strippadding}{0.1}
    \pgfmathsetmacro{\textpadding}{0.3}
    \pgfmathsetmacro{\ruleheight}{0.1}
    \providecommand{\stripfontsize}{\Huge}
    \providecommand{\captionfontsize}{\LARGE}
    \providecommand{\textfontsize}{\Large}
    \providecommand{\quotefontsize}{\small}
    \draw[line width=2mm,rounded corners=\cardroundingradius] (0,0) rectangle (\cardwidth,\cardheight);
    \draw[line width=2mm] (0,0) rectangle (\cardwidth,\cardheight);
    \node[text width=(\cardwidth-\strippadding-2*\textpadding)*1cm,below right,inner sep=0] at (\strippadding+\textpadding,\cardheight-\textpadding) 
    { 
    \begin{center} {\fontsize{80pt}{60pt}\selectfont \COW}\\\end{center}
\begin{center}
    {\captionfontsize \textsf{\textbf{SLOWER SPEED OF LIGHT}}}\end{center}
        {\textfontsize Before each turn, one chosen cow moves 1 hex in a set direction. Other cows may interact with it during their moves.}
        \tikz{\fill (0,0) rectangle (\cardwidth-2*\strippadding-2*\textpadding,\ruleheight);}\\
        {\quotefontsize \textit{}}\\[-2\baselineskip]
    };
    \node[circle,draw,text=black](c) at (.5,\cardheight-.5){};
    \node[circle,draw,text=black](c) at (\cardwidth-.5,\cardheight-.5){};
    \end{tikzpicture}%
\hspace{1pt}%
\begin{tikzpicture}
    \pgfmathsetmacro{\cardroundingradius}{5mm}
    \pgfmathsetmacro{\striproundingradius}{3mm}
    % \pgfmathsetmacro{\cardwidth}{5.9}
    % \pgfmathsetmacro{\cardheight}{9.2}
    \pgfmathsetmacro{\cardwidth}{6.1}  % Magic cards are 63x88mm
    \pgfmathsetmacro{\cardheight}{8.6}
    \pgfmathsetmacro{\stripwidth}{1.2}
    \pgfmathsetmacro{\strippadding}{0.1}
    \pgfmathsetmacro{\textpadding}{0.3}
    \pgfmathsetmacro{\ruleheight}{0.1}
    \providecommand{\stripfontsize}{\Huge}
    \providecommand{\captionfontsize}{\LARGE}
    \providecommand{\textfontsize}{\Large}
    \providecommand{\quotefontsize}{\small}
    \draw[line width=2mm,rounded corners=\cardroundingradius] (0,0) rectangle (\cardwidth,\cardheight);
    \draw[line width=2mm] (0,0) rectangle (\cardwidth,\cardheight);
    \node[text width=(\cardwidth-\strippadding-2*\textpadding)*1cm,below right,inner sep=0] at (\strippadding+\textpadding,\cardheight-\textpadding) 
    { 
    \begin{center} {\fontsize{80pt}{60pt}\selectfont \COW}\\\end{center}
\begin{center}
    {\captionfontsize \textsf{\textbf{CURVED SPACE}}}\end{center}
        {\textfontsize Outer walls are removed. Cows moving into a wall wrapp to the opposite side of the board and continue moving.}
        \tikz{\fill (0,0) rectangle (\cardwidth-2*\strippadding-2*\textpadding,\ruleheight);}\\
        {\quotefontsize \textit{}}\\[-2\baselineskip]
    };
    \node[circle,draw,text=black](c) at (.5,\cardheight-.5){};
    \node[circle,draw,text=black](c) at (\cardwidth-.5,\cardheight-.5){};
    \end{tikzpicture}%
\hspace{1pt}%
\\[-.5\lineskip]
\begin{tikzpicture}
    \pgfmathsetmacro{\cardroundingradius}{5mm}
    \pgfmathsetmacro{\striproundingradius}{3mm}
    % \pgfmathsetmacro{\cardwidth}{5.9}
    % \pgfmathsetmacro{\cardheight}{9.2}
    \pgfmathsetmacro{\cardwidth}{6.1}  % Magic cards are 63x88mm
    \pgfmathsetmacro{\cardheight}{8.6}
    \pgfmathsetmacro{\stripwidth}{1.2}
    \pgfmathsetmacro{\strippadding}{0.1}
    \pgfmathsetmacro{\textpadding}{0.3}
    \pgfmathsetmacro{\ruleheight}{0.1}
    \providecommand{\stripfontsize}{\Huge}
    \providecommand{\captionfontsize}{\LARGE}
    \providecommand{\textfontsize}{\Large}
    \providecommand{\quotefontsize}{\small}
    \draw[line width=2mm,rounded corners=\cardroundingradius] (0,0) rectangle (\cardwidth,\cardheight);
    \draw[line width=2mm] (0,0) rectangle (\cardwidth,\cardheight);
    \node[text width=(\cardwidth-\strippadding-2*\textpadding)*1cm,below right,inner sep=0] at (\strippadding+\textpadding,\cardheight-\textpadding) 
    { 
    \begin{center} {\fontsize{80pt}{60pt}\selectfont \COW}\\\end{center}
\begin{center}
    {\captionfontsize \textsf{\textbf{MAGNETIC REPULSION}}}\end{center}
        {\textfontsize Non-designated cows may stop one hex short of a blocking hex.}
        \tikz{\fill (0,0) rectangle (\cardwidth-2*\strippadding-2*\textpadding,\ruleheight);}\\
        {\quotefontsize \textit{}}\\[-2\baselineskip]
    };
    \node[circle,draw,text=black](c) at (.5,\cardheight-.5){};
    \node[circle,draw,text=black](c) at (\cardwidth-.5,\cardheight-.5){};
    \end{tikzpicture}%
\hspace{1pt}%
\begin{tikzpicture}
    \pgfmathsetmacro{\cardroundingradius}{5mm}
    \pgfmathsetmacro{\striproundingradius}{3mm}
    % \pgfmathsetmacro{\cardwidth}{5.9}
    % \pgfmathsetmacro{\cardheight}{9.2}
    \pgfmathsetmacro{\cardwidth}{6.1}  % Magic cards are 63x88mm
    \pgfmathsetmacro{\cardheight}{8.6}
    \pgfmathsetmacro{\stripwidth}{1.2}
    \pgfmathsetmacro{\strippadding}{0.1}
    \pgfmathsetmacro{\textpadding}{0.3}
    \pgfmathsetmacro{\ruleheight}{0.1}
    \providecommand{\stripfontsize}{\Huge}
    \providecommand{\captionfontsize}{\LARGE}
    \providecommand{\textfontsize}{\Large}
    \providecommand{\quotefontsize}{\small}
    \draw[line width=2mm,rounded corners=\cardroundingradius] (0,0) rectangle (\cardwidth,\cardheight);
    \draw[line width=2mm] (0,0) rectangle (\cardwidth,\cardheight);
    \node[text width=(\cardwidth-\strippadding-2*\textpadding)*1cm,below right,inner sep=0] at (\strippadding+\textpadding,\cardheight-\textpadding) 
    { 
    \begin{center} {\fontsize{80pt}{60pt}\selectfont \COW}\\\end{center}
\begin{center}
    {\captionfontsize \textsf{\textbf{ELASTIC COLLISION}}}\end{center}
        {\textfontsize After a cow collision, the stationary cow moves in the same direction that is was pushed as part of the same move.}
        \tikz{\fill (0,0) rectangle (\cardwidth-2*\strippadding-2*\textpadding,\ruleheight);}\\
        {\quotefontsize \textit{}}\\[-2\baselineskip]
    };
    \node[circle,draw,text=black](c) at (.5,\cardheight-.5){};
    \node[circle,draw,text=black](c) at (\cardwidth-.5,\cardheight-.5){};
    \end{tikzpicture}%
\hspace{1pt}%
\begin{tikzpicture}
    \pgfmathsetmacro{\cardroundingradius}{5mm}
    \pgfmathsetmacro{\striproundingradius}{3mm}
    % \pgfmathsetmacro{\cardwidth}{5.9}
    % \pgfmathsetmacro{\cardheight}{9.2}
    \pgfmathsetmacro{\cardwidth}{6.1}  % Magic cards are 63x88mm
    \pgfmathsetmacro{\cardheight}{8.6}
    \pgfmathsetmacro{\stripwidth}{1.2}
    \pgfmathsetmacro{\strippadding}{0.1}
    \pgfmathsetmacro{\textpadding}{0.3}
    \pgfmathsetmacro{\ruleheight}{0.1}
    \providecommand{\stripfontsize}{\Huge}
    \providecommand{\captionfontsize}{\LARGE}
    \providecommand{\textfontsize}{\Large}
    \providecommand{\quotefontsize}{\small}
    \draw[line width=2mm,rounded corners=\cardroundingradius] (0,0) rectangle (\cardwidth,\cardheight);
    \draw[line width=2mm] (0,0) rectangle (\cardwidth,\cardheight);
    \node[text width=(\cardwidth-\strippadding-2*\textpadding)*1cm,below right,inner sep=0] at (\strippadding+\textpadding,\cardheight-\textpadding) 
    { 
    \begin{center} {\fontsize{80pt}{60pt}\selectfont \COW}\\\end{center}
\begin{center}
    {\captionfontsize \textsf{\textbf{ENTANGLEMENT}}}\end{center}
        {\textfontsize Adjacent cows may swap places and/or move together as an entangled pair.}
        \tikz{\fill (0,0) rectangle (\cardwidth-2*\strippadding-2*\textpadding,\ruleheight);}\\
        {\quotefontsize \textit{}}\\[-2\baselineskip]
    };
    \node[circle,draw,text=black](c) at (.5,\cardheight-.5){};
    \node[circle,draw,text=black](c) at (\cardwidth-.5,\cardheight-.5){};
    \end{tikzpicture}%
\hspace{1pt}%

    \end{document}
    
