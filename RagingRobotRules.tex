
    % latexmk -pdflatex='lualatex' -pdf FontExhibition.tex
    \documentclass[parskip,landscape,letter]{scrartcl}
    \usepackage{fontspec}
    \usepackage[margin=10mm,left=20mm]{geometry}
    \usepackage{tikz}
    \usetikzlibrary{matrix,backgrounds}
    \usepackage{pifont}
    \usepackage{graphicx}
    \usepackage{chessfss}
    \usepackage{setspace}
    \usepackage{enumitem}
    \usepackage{amssymb}
    \usepackage{graphicx,calc}
    \graphicspath{{./graphics/}}
    \usepackage{graphicx,calc,amssymb}
\newlength\myheight
\newlength\mydepth
\settototalheight\myheight{Xygp}
\settodepth\mydepth{Xygp}
\setlength\fboxsep{0pt}
\newcommand*\inlinegraphics[1]{%
  \settototalheight\myheight{Xygp}%
  \settodepth\mydepth{Xygp}%
  \raisebox{-.1\mydepth}{\includegraphics[height=.85\myheight]{#1}}%
}
\providecommand{\WKnight}{♘}
\providecommand{\BKnight}{♞}
\providecommand{\Knight}{\BKnight}
\providecommand{\WPawn}{♙}
\providecommand{\BPawn}{♟︎}
\providecommand{\Pawn}{\BPawn}
\providecommand{\WKing}{♔}
\providecommand{\BKing}{♚}
\providecommand{\King}{\BKing}
\providecommand{\WQueen}{♕}
\providecommand{\BQueen}{♛}
\providecommand{\Queen}{\BQueen}
\providecommand{\WBishop}{♗}
\providecommand{\BBishop}{♝}
\providecommand{\Bishop}{\BBishop}
\providecommand{\WRook}{♖}
\providecommand{\BRook}{♜}
\providecommand{\Rook}{\BRook}
\providecommand{\All}{\ding{72}}
\providecommand{\KMI}{➊}
% \providecommand{\KII}{➋}
% \providecommand{\KIII}{➌}
% \providecommand{\KIV}{➍} 
\providecommand{\KI}{➀}
\providecommand{\KII}{➁}
\providecommand{\KIII}{➂}
\providecommand{\KIV}{➃}
\providecommand{\RI}{\inlinegraphics{robot1.png}}
\providecommand{\RII}{\inlinegraphics{robot2.png}}
\providecommand{\RIII}{\inlinegraphics{robot3.png}}
\providecommand{\RIV}{\inlinegraphics{robot4.png}}
\providecommand{\RV}{\inlinegraphics{robot5.png}}
\providecommand{\Move}{\inlinegraphics{move.png}}
\providecommand{\Attack}{\inlinegraphics{attack.png}}
\providecommand{\Drop}{\inlinegraphics{down.pdf}}
\providecommand{\Same}{\inlinegraphics{identical.png}}
\providecommand{\Jump}{\inlinegraphics{jump.png}}
\providecommand{\Back}{\inlinegraphics{back.png}}
\providecommand{\Board}{\inlinegraphics{ChessBoard.png}}
\providecommand{\Sym}{\fontsize{20pt}{15pt}\selectfont}
\providecommand{\NoKing}{\inlinegraphics{NOKING}}
\providecommand{\NoPawn}{\inlinegraphics{NOPAWN}}
\providecommand{\NoKnight}{\inlinegraphics{NOKNIGHT}}
\providecommand{\NoQueen}{\inlinegraphics{NOQUEEN}}
\providecommand{\NoRook}{\inlinegraphics{NOROOK}}
\providecommand{\NoBishop}{\inlinegraphics{NOBISHOP}}
\providecommand{\Action}{\inlinegraphics{action.png}}
\providecommand{\TenEight}{\inlinegraphics{10x8.pdf}}
\providecommand{\EightTen}{\inlinegraphics{8x10.pdf}}
\providecommand{\PM}{±}
\providecommand{\diff}{⇌}


    \begin{document}
    \setmainfont[Extension={.ttf},ItalicFont={DejaVuSerif-Italic}]{FreeSerif}
    
\begin{tikzpicture}
    \pgfmathsetmacro{\cardroundingradius}{5mm}
    \pgfmathsetmacro{\striproundingradius}{3mm}
    % \pgfmathsetmacro{\cardwidth}{5.9}
    % \pgfmathsetmacro{\cardheight}{9.2}
    \pgfmathsetmacro{\cardwidth}{6.1}  % Magic cards are 63x88mm
    \pgfmathsetmacro{\cardheight}{8.6}
    \pgfmathsetmacro{\stripwidth}{1.2}
    \pgfmathsetmacro{\strippadding}{0.1}
    \pgfmathsetmacro{\textpadding}{0.3}
    \pgfmathsetmacro{\ruleheight}{0.1}
    \providecommand{\stripfontsize}{\Huge}
    \providecommand{\captionfontsize}{\LARGE}
    \providecommand{\textfontsize}{\Large}
    \providecommand{\quotefontsize}{\small}
    \draw[line width=2mm,rounded corners=\cardroundingradius] (0,0) rectangle (\cardwidth,\cardheight);
    \draw[line width=2mm] (0,0) rectangle (\cardwidth,\cardheight);
    \node[text width=(\cardwidth-\strippadding-2*\textpadding)*1cm,below right,inner sep=0] at (\strippadding+\textpadding,\cardheight-\textpadding) 
    { 
    \begin{center} {\fontsize{80pt}{60pt}\selectfont \RI}\\\end{center}
\begin{center}
    {\captionfontsize \textsf{\textbf{VERTICAL STACK}}}\end{center}
        {\textfontsize Lay boards in a row. You may use a move to go up or down a board in the same position.}
        \tikz{\fill (0,0) rectangle (\cardwidth-2*\strippadding-2*\textpadding,\ruleheight);}\\
        {\quotefontsize \textit{}}\\[-2\baselineskip]
    };
    \node[circle,draw,text=black](c) at (.5,\cardheight-.5){C};
    \node[circle,draw,text=black](c) at (\cardwidth-.5,\cardheight-.5){+};
    \end{tikzpicture}%
\hspace{1pt}%
\begin{tikzpicture}
    \pgfmathsetmacro{\cardroundingradius}{5mm}
    \pgfmathsetmacro{\striproundingradius}{3mm}
    % \pgfmathsetmacro{\cardwidth}{5.9}
    % \pgfmathsetmacro{\cardheight}{9.2}
    \pgfmathsetmacro{\cardwidth}{6.1}  % Magic cards are 63x88mm
    \pgfmathsetmacro{\cardheight}{8.6}
    \pgfmathsetmacro{\stripwidth}{1.2}
    \pgfmathsetmacro{\strippadding}{0.1}
    \pgfmathsetmacro{\textpadding}{0.3}
    \pgfmathsetmacro{\ruleheight}{0.1}
    \providecommand{\stripfontsize}{\Huge}
    \providecommand{\captionfontsize}{\LARGE}
    \providecommand{\textfontsize}{\Large}
    \providecommand{\quotefontsize}{\small}
    \draw[line width=2mm,rounded corners=\cardroundingradius] (0,0) rectangle (\cardwidth,\cardheight);
    \draw[line width=2mm] (0,0) rectangle (\cardwidth,\cardheight);
    \node[text width=(\cardwidth-\strippadding-2*\textpadding)*1cm,below right,inner sep=0] at (\strippadding+\textpadding,\cardheight-\textpadding) 
    { 
    \begin{center} {\fontsize{80pt}{60pt}\selectfont \RII}\\\end{center}
\begin{center}
    {\captionfontsize \textsf{\textbf{SIMD}}}\end{center}
        {\textfontsize The same instruction is given to all robots.}
        \tikz{\fill (0,0) rectangle (\cardwidth-2*\strippadding-2*\textpadding,\ruleheight);}\\
        {\quotefontsize \textit{}}\\[-2\baselineskip]
    };
    \node[circle,draw,text=black](c) at (.5,\cardheight-.5){C};
    \node[circle,draw,text=black](c) at (\cardwidth-.5,\cardheight-.5){+};
    \end{tikzpicture}%
\hspace{1pt}%
\begin{tikzpicture}
    \pgfmathsetmacro{\cardroundingradius}{5mm}
    \pgfmathsetmacro{\striproundingradius}{3mm}
    % \pgfmathsetmacro{\cardwidth}{5.9}
    % \pgfmathsetmacro{\cardheight}{9.2}
    \pgfmathsetmacro{\cardwidth}{6.1}  % Magic cards are 63x88mm
    \pgfmathsetmacro{\cardheight}{8.6}
    \pgfmathsetmacro{\stripwidth}{1.2}
    \pgfmathsetmacro{\strippadding}{0.1}
    \pgfmathsetmacro{\textpadding}{0.3}
    \pgfmathsetmacro{\ruleheight}{0.1}
    \providecommand{\stripfontsize}{\Huge}
    \providecommand{\captionfontsize}{\LARGE}
    \providecommand{\textfontsize}{\Large}
    \providecommand{\quotefontsize}{\small}
    \draw[line width=2mm,rounded corners=\cardroundingradius] (0,0) rectangle (\cardwidth,\cardheight);
    \draw[line width=2mm] (0,0) rectangle (\cardwidth,\cardheight);
    \node[text width=(\cardwidth-\strippadding-2*\textpadding)*1cm,below right,inner sep=0] at (\strippadding+\textpadding,\cardheight-\textpadding) 
    { 
    \begin{center} {\fontsize{80pt}{60pt}\selectfont \RIII}\\\end{center}
\begin{center}
    {\captionfontsize \textsf{\textbf{MAYBE(SIMD)}}}\end{center}
        {\textfontsize The same instruction may be given to all robots.}
        \tikz{\fill (0,0) rectangle (\cardwidth-2*\strippadding-2*\textpadding,\ruleheight);}\\
        {\quotefontsize \textit{}}\\[-2\baselineskip]
    };
    \node[circle,draw,text=black](c) at (.5,\cardheight-.5){C};
    \node[circle,draw,text=black](c) at (\cardwidth-.5,\cardheight-.5){+};
    \end{tikzpicture}%
\hspace{1pt}%
\begin{tikzpicture}
    \pgfmathsetmacro{\cardroundingradius}{5mm}
    \pgfmathsetmacro{\striproundingradius}{3mm}
    % \pgfmathsetmacro{\cardwidth}{5.9}
    % \pgfmathsetmacro{\cardheight}{9.2}
    \pgfmathsetmacro{\cardwidth}{6.1}  % Magic cards are 63x88mm
    \pgfmathsetmacro{\cardheight}{8.6}
    \pgfmathsetmacro{\stripwidth}{1.2}
    \pgfmathsetmacro{\strippadding}{0.1}
    \pgfmathsetmacro{\textpadding}{0.3}
    \pgfmathsetmacro{\ruleheight}{0.1}
    \providecommand{\stripfontsize}{\Huge}
    \providecommand{\captionfontsize}{\LARGE}
    \providecommand{\textfontsize}{\Large}
    \providecommand{\quotefontsize}{\small}
    \draw[line width=2mm,rounded corners=\cardroundingradius] (0,0) rectangle (\cardwidth,\cardheight);
    \draw[line width=2mm] (0,0) rectangle (\cardwidth,\cardheight);
    \node[text width=(\cardwidth-\strippadding-2*\textpadding)*1cm,below right,inner sep=0] at (\strippadding+\textpadding,\cardheight-\textpadding) 
    { 
    \begin{center} {\fontsize{80pt}{60pt}\selectfont \RIV}\\\end{center}
\begin{center}
    {\captionfontsize \textsf{\textbf{SILVER SURFER}}}\end{center}
        {\textfontsize The silver robot must touch another robot that moved.}
        \tikz{\fill (0,0) rectangle (\cardwidth-2*\strippadding-2*\textpadding,\ruleheight);}\\
        {\quotefontsize \textit{}}\\[-2\baselineskip]
    };
    \node[circle,draw,text=black](c) at (.5,\cardheight-.5){C};
    \node[circle,draw,text=black](c) at (\cardwidth-.5,\cardheight-.5){+};
    \end{tikzpicture}%
\hspace{1pt}%
\\[-.5\lineskip]
\begin{tikzpicture}
    \pgfmathsetmacro{\cardroundingradius}{5mm}
    \pgfmathsetmacro{\striproundingradius}{3mm}
    % \pgfmathsetmacro{\cardwidth}{5.9}
    % \pgfmathsetmacro{\cardheight}{9.2}
    \pgfmathsetmacro{\cardwidth}{6.1}  % Magic cards are 63x88mm
    \pgfmathsetmacro{\cardheight}{8.6}
    \pgfmathsetmacro{\stripwidth}{1.2}
    \pgfmathsetmacro{\strippadding}{0.1}
    \pgfmathsetmacro{\textpadding}{0.3}
    \pgfmathsetmacro{\ruleheight}{0.1}
    \providecommand{\stripfontsize}{\Huge}
    \providecommand{\captionfontsize}{\LARGE}
    \providecommand{\textfontsize}{\Large}
    \providecommand{\quotefontsize}{\small}
    \draw[line width=2mm,rounded corners=\cardroundingradius] (0,0) rectangle (\cardwidth,\cardheight);
    \draw[line width=2mm] (0,0) rectangle (\cardwidth,\cardheight);
    \node[text width=(\cardwidth-\strippadding-2*\textpadding)*1cm,below right,inner sep=0] at (\strippadding+\textpadding,\cardheight-\textpadding) 
    { 
    \begin{center} {\fontsize{80pt}{60pt}\selectfont \RI}\\\end{center}
\begin{center}
    {\captionfontsize \textsf{\textbf{180}}}\end{center}
        {\textfontsize Rotate all boards 180 degrees.}
        \tikz{\fill (0,0) rectangle (\cardwidth-2*\strippadding-2*\textpadding,\ruleheight);}\\
        {\quotefontsize \textit{}}\\[-2\baselineskip]
    };
    \node[circle,draw,text=black](c) at (.5,\cardheight-.5){C};
    \node[circle,draw,text=black](c) at (\cardwidth-.5,\cardheight-.5){+};
    \end{tikzpicture}%
\hspace{1pt}%
\begin{tikzpicture}
    \pgfmathsetmacro{\cardroundingradius}{5mm}
    \pgfmathsetmacro{\striproundingradius}{3mm}
    % \pgfmathsetmacro{\cardwidth}{5.9}
    % \pgfmathsetmacro{\cardheight}{9.2}
    \pgfmathsetmacro{\cardwidth}{6.1}  % Magic cards are 63x88mm
    \pgfmathsetmacro{\cardheight}{8.6}
    \pgfmathsetmacro{\stripwidth}{1.2}
    \pgfmathsetmacro{\strippadding}{0.1}
    \pgfmathsetmacro{\textpadding}{0.3}
    \pgfmathsetmacro{\ruleheight}{0.1}
    \providecommand{\stripfontsize}{\Huge}
    \providecommand{\captionfontsize}{\LARGE}
    \providecommand{\textfontsize}{\Large}
    \providecommand{\quotefontsize}{\small}
    \draw[line width=2mm,rounded corners=\cardroundingradius] (0,0) rectangle (\cardwidth,\cardheight);
    \draw[line width=2mm] (0,0) rectangle (\cardwidth,\cardheight);
    \node[text width=(\cardwidth-\strippadding-2*\textpadding)*1cm,below right,inner sep=0] at (\strippadding+\textpadding,\cardheight-\textpadding) 
    { 
    \begin{center} {\fontsize{80pt}{60pt}\selectfont \RII}\\\end{center}
\begin{center}
    {\captionfontsize \textsf{\textbf{HOLES}}}\end{center}
        {\textfontsize Mark a hole on each board. Robots may not pass over holes.}
        \tikz{\fill (0,0) rectangle (\cardwidth-2*\strippadding-2*\textpadding,\ruleheight);}\\
        {\quotefontsize \textit{}}\\[-2\baselineskip]
    };
    \node[circle,draw,text=black](c) at (.5,\cardheight-.5){C};
    \node[circle,draw,text=black](c) at (\cardwidth-.5,\cardheight-.5){+};
    \end{tikzpicture}%
\hspace{1pt}%
\begin{tikzpicture}
    \pgfmathsetmacro{\cardroundingradius}{5mm}
    \pgfmathsetmacro{\striproundingradius}{3mm}
    % \pgfmathsetmacro{\cardwidth}{5.9}
    % \pgfmathsetmacro{\cardheight}{9.2}
    \pgfmathsetmacro{\cardwidth}{6.1}  % Magic cards are 63x88mm
    \pgfmathsetmacro{\cardheight}{8.6}
    \pgfmathsetmacro{\stripwidth}{1.2}
    \pgfmathsetmacro{\strippadding}{0.1}
    \pgfmathsetmacro{\textpadding}{0.3}
    \pgfmathsetmacro{\ruleheight}{0.1}
    \providecommand{\stripfontsize}{\Huge}
    \providecommand{\captionfontsize}{\LARGE}
    \providecommand{\textfontsize}{\Large}
    \providecommand{\quotefontsize}{\small}
    \draw[line width=2mm,rounded corners=\cardroundingradius] (0,0) rectangle (\cardwidth,\cardheight);
    \draw[line width=2mm] (0,0) rectangle (\cardwidth,\cardheight);
    \node[text width=(\cardwidth-\strippadding-2*\textpadding)*1cm,below right,inner sep=0] at (\strippadding+\textpadding,\cardheight-\textpadding) 
    { 
    \begin{center} {\fontsize{80pt}{60pt}\selectfont \RII}\\\end{center}
\begin{center}
    {\captionfontsize \textsf{\textbf{TELEPORT}}}\end{center}
        {\textfontsize A robot may teleport between any two locations of the same color as a move.}
        \tikz{\fill (0,0) rectangle (\cardwidth-2*\strippadding-2*\textpadding,\ruleheight);}\\
        {\quotefontsize \textit{}}\\[-2\baselineskip]
    };
    \node[circle,draw,text=black](c) at (.5,\cardheight-.5){C};
    \node[circle,draw,text=black](c) at (\cardwidth-.5,\cardheight-.5){+};
    \end{tikzpicture}%
\hspace{1pt}%
\begin{tikzpicture}
    \pgfmathsetmacro{\cardroundingradius}{5mm}
    \pgfmathsetmacro{\striproundingradius}{3mm}
    % \pgfmathsetmacro{\cardwidth}{5.9}
    % \pgfmathsetmacro{\cardheight}{9.2}
    \pgfmathsetmacro{\cardwidth}{6.1}  % Magic cards are 63x88mm
    \pgfmathsetmacro{\cardheight}{8.6}
    \pgfmathsetmacro{\stripwidth}{1.2}
    \pgfmathsetmacro{\strippadding}{0.1}
    \pgfmathsetmacro{\textpadding}{0.3}
    \pgfmathsetmacro{\ruleheight}{0.1}
    \providecommand{\stripfontsize}{\Huge}
    \providecommand{\captionfontsize}{\LARGE}
    \providecommand{\textfontsize}{\Large}
    \providecommand{\quotefontsize}{\small}
    \draw[line width=2mm,rounded corners=\cardroundingradius] (0,0) rectangle (\cardwidth,\cardheight);
    \draw[line width=2mm] (0,0) rectangle (\cardwidth,\cardheight);
    \node[text width=(\cardwidth-\strippadding-2*\textpadding)*1cm,below right,inner sep=0] at (\strippadding+\textpadding,\cardheight-\textpadding) 
    { 
    \begin{center} {\fontsize{80pt}{60pt}\selectfont \RIV}\\\end{center}
\begin{center}
    {\captionfontsize \textsf{\textbf{NINJAS}}}\end{center}
        {\textfontsize You may use a move to pass through a wall.}
        \tikz{\fill (0,0) rectangle (\cardwidth-2*\strippadding-2*\textpadding,\ruleheight);}\\
        {\quotefontsize \textit{}}\\[-2\baselineskip]
    };
    \node[circle,draw,text=black](c) at (.5,\cardheight-.5){C};
    \node[circle,draw,text=black](c) at (\cardwidth-.5,\cardheight-.5){+};
    \end{tikzpicture}%
\hspace{1pt}%
\\[-.5\lineskip]

    \end{document}
    
